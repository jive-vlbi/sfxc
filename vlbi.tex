\section{VLBI}\label{sec:vlbi}
\acomment{NGHK: Move to the introduction}{Recent Astronomical research
  studies the deep-sky (sources farther and farther away from us)
  which requires higher and higher angular resolution to capture all
  the details of the observed sources.  Increasing the size of a
  telescope dish increases the angular resolution of the image.
  However, it is difficult to build a moveable telescope dish with a
  size much larger than 100m.} Interferometry provides a possible
solution to \acomment{change}{this problem} as it combines the measurements of
several telescopes to simulate a dish of a size equivalent to the
maximal distance between the farthest telescopes on the plane
orthogonal to the viewing direction. Numerous arrays (groups of
telescopes) use this technique, e.g., the VLA (Very Large Array),
Lofar (Low-frequency array), the EVN (European VLBI Network) or the
VLBA (Very Large Baseline Array). Interferometry with telescopes that
are geographically very far apart is refered to as Very Large Baseline
Interferometry (VLBI). VLBI permits to build a virtual radio-telescope
with a dish of a size of the Earth. As the angular resolution of a
VLBI experiment depends on the maximal projected distance between two
radio-telescopes, VLBI achieves unsurpassed angular resolution with
the drawback of a relatively low sensitivity. Sensitivity is important
as it allows to detect fainter astronomical objects, increasing the
sensitivity is possible by adding more radio-telescopes or by
increasing the sample's resolution or the sampling rate, and thus
having more data gathered per telescope.

In order to get the final picture the signal gathered from the
radio-telescopes have to be correlated at a central place, the Joint
Institute for VLBI in Europe (JIVE), for correlation.  JIVE is
operating a dedicated hardware correlator~\cite{EVNCorrelator}.

The maximal capacity of this hardware correlator is 16 telescopes at a
data rate of 1Gbs each. The requirements on both the data streams and
the computing power are shown in Table~\ref{tab:speed}.

% \marginpar{NGHK: Check 16Mb/s in table}
\begin{table}
  \centering
  \begin{tabular}[c]{|l|l|l|l|l|l|}
    \hline
    Description & \# & \#  & data-rate & spect/prod & Tflops\\
    & telescopes & sub-bands & (Mb/s) &  & \\
    \hline
    \hline
    Fabric-demo &4 &2 &16 &32 &0.16\\
    1 Gb/s, full array  &16 &16 &1024 &16 &83.39\\
    future VLBI &32 &32 &4096 &256 &\verb|~|21457\\
    \hline
  \end{tabular}
  \caption{Network bandwidths and computing power needed for an {\it e}-VLBI
    experiment based on a XF architecture.}
  \label{tab:speed}
\end{table}

\paragraph{Correlation}\marginpar{NGHK: Split how/why correlation}
Correlation is the process by which data from multiple telescopes is
collected and combined to measure the spatial Fourier components of
the image of the sky.

Assume that we are correlating the signal of two telescopes. First,
both signals are delayed to account for the different time at which
the signal arrives at the telescopes, see
Figure~\ref{fig:correlation_diagram}. This process requires very
accurate timing information in the data and a very detailed model of
the geometry of the experiment. After the application of delay the
signals are properly aligned and the can be correlated.
Correlation~\cite{def_correlation} is mathematically defined as a
function on two signals in which \acomment{this is the mathematical
  definition}{the first signal is delayed with discrete steps and the
  integral is computed of the delayed signal multiplied with the
  second signal}. To increase the signal to noise ratio, the
correlated signal is averaged over a certain period of time. Typical
averaging times lie in the range of $0.25-4$ seconds.

For more than two stations, each station is correlated with itself
(auto-correlation) and every other station (cross-correlation). Note
that the complexity of the correlation is quadratic in the number of
telescopes, as it is linear in the number of baselines (telescope
pairs).

\paragraph{{\it e}-VLBI}
Traditionally, in VLBI, the data is recorded at the telescopes on disk
packs during an experiment. After the experiment the disks are shipped
to a central institute. There can be several weeks between the
experiment and the time when the correlated data becomes available.

Currently, JIVE is in the transition phase from traditional VLBI to
{\it e}-VLBI~\cite{szomoru-2004}. In an electronic VLBI ({\it e}-VLBI)
experiment, data from the telescopes is transferred directly over the
internet to JIVE, where it is streamed into the correlator in real
time. The data transport from the telescopes to JIVE goes over several
networks like local connections, paths provided by NRENs and the
G\'EANT backbone in Europe.

Transporting the data over the network has several advantages over a
traditional experiment. Obviously, the results of the experiments are
almost immediately available. This opens up the possibility to change
the course of an experiment based on earlier findings. Also, {\it
  e}-VLBI allows for real time analysis of the data and helps to
identify and resolve minor technical problems in the data collection
during the experiment.

Several experiments in the past have shown that real time {\it e}-VLBI
is possible. The EC funds the EXPReS project\footnote{EXPReS is made
  possible through the support of the European Commission (DG-INFSO),
  Sixth Framework Programme, Contract \#026642.}~\cite{EXPReS} which
aims at building a production-level {\it e}-VLBI instrument of upto 16
intercontinental telescopes connected in real-time to JIVE and
available to the general astronomy community.

%%% Local Variables:
%%% mode: latex
%%% TeX-master: "Ingrid"
%%% End:
