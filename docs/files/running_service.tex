\section{Running the web service}\label{running}

Since we are going to copy the data over a grid ftp we need to set up and
intiate the grid ftp server first. Assumed that Globus is installed:

\begin{itemize}
\item log in as jops (to i.e. huygens) and set up the Globus environment with
either: \\
  \verb|  $ setenv GLOBUS_LOCATION /huygens_1/jops/globus| \\
  \verb|  $ source $GLOBUS_LOCATION/etc/globus-user-env.csh| \\
or: \\
  \verb|  $ export GLOBUS_LOCATION=/huygens_1/jops/globus| \\
  \verb|  $ . $GLOBUS_LOCATION/etc/globus-user-env.sh|
\item Now, you can start the server (as user jops) by: \\
  \verb|  $ grid-proxy-init -key /etc/grid-security/containerkey.pem -cert \|\\
  \verb|    /etc/grid-security/containercert.pem| \\
  \verb|  $ globus-gridftp-server -p 2811|
\item After that, you can log in as yourself and set up the environment as
above.  Make sure that you have a ~/.globus directory with your
private key (userkey.pem) and your certificate.(usercert.pem, which
you got by mail).  If that's the case, you should be able to create a
proxy certificate using: \\
  \verb|  $ grid-proxy-init| \\
This will ask you to enter the passphrase for your key.
\item You should be able to use the GridFTP client:\\
  \verb|  $ globus-url-copy gsiftp://huygens.nfra.nl/SRC file://DEST|
\item Now you are ready to run the web service:
  \verb|  python TranslationNodeService.py|
\end{itemize}

More documentation can be found at:\\
  \verb|  http://www.globus.org/toolkit/docs/4.0/data/gridftp/|





