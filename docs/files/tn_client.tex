\subsection{Client for Translation Node web service}\label{tn_client}

We should write the necessary consumer code. In this case the client needs to
send data to the service. For the complete client and service codes please see
the original codes. Below we give a more compact generic forms of these codes.

\begin{verbatim}
TranslationNodeRequest.py:

# import the generated class stubs
from TranslationNode_services import *

# create a new request
req = startTranslationJobMessage()
req.Param0 = req.new_param0()

# of course we need to read in (or define) data to be sent first
# details can be seen from the original code
req.Param0.BrokerIPAddress=brokerIPAdress
req.Param0.ChunkSize=chunkSize
req.Param0.StartTime=startTime
req.Param0.EndTime=endTime
req.Param0.StationName=stationName
req.Param0.ExperimentName=experimentName

# get a port proxy instance
# translationnode is the name of the service
# portnumber is defined in the input file
loc = TranslationNodeLocator()
serviceLocation = 'http://huygens:' + portNumber + '/translationnode'
port = loc.getTranslationNodePortType(serviceLocation)

# actualy ask the service to do the job
resp = port.startTranslationJob(req)

# print out the response
print resp
\end{verbatim}

An example of the data file is as follows:

\begin{verbatim}
experiment_data.inp:

# grid broker IP address
# where the response notification will be sent
jop32
# chunk size in bytes or "scan size"
# when "scan size", calculates the actual chunk size as the size of a scan 
10000
# port number
8082
# start time of the first chunk
2007y158d18h41m30s
# end time
2007y158d18h41m32s
# Station code (two letter name of station)
Ef
# vax file name
ez015.vix
\end{verbatim}

