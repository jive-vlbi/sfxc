\section{Future work}\label{sec:conclusion}

Within Future Arrays of Broadband Radio-telescopes on Internet
Computing (FABRIC), which is a joint research activity in the EXPReS
project, we are currently implementing the software correlator. The
work flow management is designed by the the Polish Supercomputing and
Networking Center in Poznan.

In order to be able to guarantee data transfer, it might be necessary
to add an extra node near the telescopes with guaranteed bandwidth to
the telescope. This node could buffer the input data in case the
transfer rates to the Grid drop temporarily. When the software
correlator is run on a cluster of super computers, this node can also
send large time slices directly to alternating super computers. In
this way the network connectivity is optimized by introducing an
additional layer of input nodes.

Other research areas that need attention are found in resource
leveling, dealing with delays in data transfers and managing
insufficient computing power.

We also want to test several networking protocols to optimize the data
transfer. For example, we could imagine using UDP for the main data
stream, where some data loss is acceptable and a TCP connection for
data-headers to guarantee their proper delivery.

%%% Local Variables:
%%% mode: latex
%%% TeX-master: "Ingrid"
%%% End:
