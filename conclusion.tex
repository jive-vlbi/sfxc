\section{Future work}\label{sec:futurework}

\subsection{Future works}

\subsubsection{Related to EXPRESS and e-VLBI}
Within Future Arrays of Broadband Radio-telescopes on Internet
Computing (FABRIC), which is a joint research activity in the EXPReS
project, we are currently implementing the software correlator. The
work flow management is designed by the the Polish Supercomputing and
Networking Center in Poznan.

In order to be able to guarantee data transfer, it might be necessary
to add an extra node near the telescopes with guaranteed bandwidth to
the telescope. This node could buffer the input data in case the
transfer rates to the Grid drop temporarily. When the software
correlator is run on a cluster of super computers, this node can also
send large time slices directly to alternating super computers. In
this way the network connectivity is optimized by introducing an
additional layer of input nodes.

\subsubsection{SCARIe specific ?}
\begin{itemize}
\item real-time runing: Testing the traffic isolation features of Starplane... ie other big job running 
in DAS-3 will not slow down sfxc and thus permit to have confidences 
that a real-time experiment will not have to be cancel due to 
buffer-overflow. It is also interresting to look which amount of "extra" ressource
may be requested in case some reserved/allocated one fails during an experiment. 
How can we provide a good level of fallback mechanisme in the context of real-time ?
\item non real-time: Other research areas that need attention are found in resource
leveling, dealing with delays in data transfers and managing
insufficient computing power.
\item technical aspects: We also want to test several networking protocols to optimize the data
transfer. For example, we could imagine using UDP for the main data
stream, where some data loss is acceptable and a TCP connection for
data-headers to guarantee their proper delivery.
\end{itemize}


\section{Conclusion}
During the first year we have laid the foundation for a flexible software correlator based on 
the distributed computing technologies....
Lorem ipsum dolor sit amet, consectetuer
adipiscing elit. Ut purus elit, vestibulum ut, placerat ac,
adipiscing vitae, felis. Curabitur dictum gravida mauris. Nam arcu
libero, nonummy eget, consectetuer id, vulputate a, magna. Donec
vehicula augue eu neque. Pellentesque habitant morbi tristique
senectus et netus et malesuada fames ac turpis egestas. Mauris ut
leo. Cras viverra metus rhoncus sem. Nulla et lectus vestibulum urna
fringilla ultrices. Phasellus eu tellus sit amet tortor gravida
placerat. Integer sapien est, iaculis in, pretium quis, viverra ac,
nunc. Praesent eget sem vel leo ultrices bibendum. Aenean faucibus.
Morbi dolor nulla, malesuada eu, pulvinar at, mollis ac, nulla.
Curabitur auctor semper nulla. Donec varius orci eget risus. 

\section{Acknowledgment}
\scarie\ is a joint research project of JIVE, the UvA and SARA funded by the Netherlands Organization for Scientific Research (NWO). 


