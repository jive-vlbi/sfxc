\section{Future work}\label{sec:futurework}

\subsection{Future works}

\subsubsection{Software correlation}
Within \scarie\ and a related project FABRIC, which is a joint
research activity in the EXPR{\it e}S project, we are currently
implementing the software correlator.

For the real-time software correlator, it might be necessary to add an
extra node near the telescopes with guaranteed bandwidth to the
telescope. This node could buffer the input data in case the transfer
rates to the Grid drop temporarily. When the software correlator is
run at different geographic locations, this node can also send large
time slices directly to alternating super computers. In this way the
network requirements might be lowered.

\subsubsection{DAS-3 and StarPlane related.}
\begin{itemize}
\item real-time runing: Testing the traffic isolation features of
  Starplane... ie other big job running in DAS-3 will not slow down
  sfxc and thus permit to have confidences that a real-time experiment
  will not have to be cancel due to a lack of resources. It is also
  interresting to look which amount of "extra" ressource may be
  requested in case some reserved/allocated one fails during an
  experiment.  How can we provide a good level of fallback mechanisms
  in the context of real-time ?
\item non real-time: Other research areas that need attention are
  found in resource leveling, dealing with delays in data transfers
  and managing insufficient computing power.
\item technical aspects: We also want to test several networking
  protocols to optimize the data transfer. For example, we could
  imagine using UDP for the main data stream, where some data loss is
  acceptable and a TCP connection for data-headers to guarantee their
  proper delivery.
\end{itemize}


\section{Conclusion}
During the first year we have laid the foundation for a
flexible software correlator based on distributed
computing technologies. Using this software correlator we
are currently collaborating intensively with the StarPlane
project in order to test incoming network with guarenteed 
Quality of Services for grid. 

\section{Acknowledgment}
\scarie\ is a joint research project of JIVE, the UvA and SARA funded by the Netherlands Organization for Scientific Research (NWO). 


