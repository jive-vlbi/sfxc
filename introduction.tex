\section{Introduction}
\begin{comment}
  Very Long Baseline Interferometry (VLBI) \cite{VLBIbook} is a type of
  interferometry used in radio astronomy, in which data received at
  several telescopes is combined to produce an image with very high
  resolution. VLBI can be used for both astronomy and geodesy.  For
  astronomy, VLBI provides high-resolution images of radio sources in
  the sky, whereas in geodesy VLBI measures the location of the
  telescopes and the Earth Orientation Parameters (EOP).
\end{comment}

Recent Astronomical research studies the deep-sky (sources farther and
farther away from us) which requires higher and higher angular
resolution to capture all the details of the observed sources.
Increasing the size of a telescope dish increases the angular
resolution of the image.  However, due to mechanical constraints, it
is difficult to build moveable telescope dishes with a size much
larger than 100m. Interferometry provides a possible solution to this
problem as it permits to combine the measurements of several
telescopes to simulate a dish of a size equivalent to the maximal
distance between the farthest telescopes on the plane orthogonal to
the viewing direction.  This approach is called Very Large Baseline
Interferometry (VLBI) and permits us to build a virtual telescope with
a dish of a size of the earth (or larger). Once the data has been
recorded, the data of each pair of stations has to be correlated.  In
the \scarie\ project we are developing and analyzing the capabilities
of making a software correlator that uses the processing power of a
grid.

\scarie\ is a typical example of a recent trend of the e-Science 
community in which computation resources and scientific instruments 
are connected worldwide through high-speed networks. We think that to 
generalize the utilization of such world-size inter-connected facility, 
the grids and their middlewares have to offer services that matches the three
following user-application requirements:
\begin{itemize}
\item \emph{better performances}: an application is said performance
  limited if the resources needed to run it are not large enough to
  satisfy its needs. Performance limitations can have multiple
  origins: insufficient computing resources, memory, high network
  latency or low network throughput.
  
\item \emph{isolated environment}: there are classes of applications
  that can only be executed if put in an isolated environment in the 
  sense that the other users'application cannot interfere. Examples are 
  real-time distributed application, or benchmarks that require reproducible
  result. 

\item \emph{scheduling}: some application requires to be synchronized
  with "external events" (like having radio-telescope looking at a
  specific location in the sky at a specific date). In order to
  execute such application it is mandatory to schedule its execution 
  and reserve the needed resource in advance.
\end{itemize}
In order to fully reach the \scarie\ objectives these three
requirements have to be addressed for networking resource, nodes and
data space. These are still hot-topics in the grid community,
especially because networking has a long history as a
\emph{best-effort} shared resourced which is incompatible with an
isolated network environment. For this reason we are conducting our
experiments using the experimental \das3\ grids as it have an
user-controllable dynamic photonic network called StarPlane to build,
on demand and application specific, isolated virtual-network on top of
the complete grid. A such service would permit to run \scarie\ with a
good confidence level that a real-time
experiment will not be disturbed by other users. \\

The rest of the paper is organized as follows. Section~\ref{sec:vlbi}
contains a general introduction to VLBI and its recent development 
called \evlbi. Section~\ref{sec:softwarecorrelation} describes the 
architecture of the software correlator. Section~\ref{sec:network} 
describes how the software correlator make use of the StarPlane services 
and the \das3 resources. We present benchmarks in Section~\ref{sec:benchmarks} 
and conclude the paper with future work in Section~\ref{sec:conclusion}.




%%% Local Variables:
%%% mode: latex
%%% TeX-master: "Ingrid"
%%% End:
