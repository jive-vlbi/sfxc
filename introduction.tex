\section{Introduction}
Very Long Baseline Interferometry (VLBI) \cite{VLBIbook} is a type of
interferometry used in radio astronomy, in which data received at
several telescopes is combined to produce an image with very high
resolution. 
%\acomment{Remove.}{VLBI can be used for both astronomy and geodesy.  For
%astronomy, VLBI provides high-resolution images of radio sources in
%the sky, whereas in geodesy VLBI measures the location of the
%telescopes and the Earth Orientation Parameters (EOP).}
Recent Astronomical research studies the deep-sky (sources farther and
farther away from us) which requires higher and higher angular
resolution to capture all the details of the observed sources.
Increasing the size of a telescope dish increases the angular
resolution of the image. Due ue to mechanical constraints, it
is difficult to build moveable telescope dishes with a size much
larger than 100m. With VLBI, it is possible simulate a dish of a size 
equivalent to the maximal distance between the farthest telescopes \acomment{This details can go 
to 2}{on the plane orthogonal to
the viewing direction} so making virtual telescope with
a dish of a size of the Earth (or larger). The data is recorded
and sent to a central facility for processing, the central part of this
processing imply is pairwise signal correlation. In
the \scarie\ project we are developing and analyzing the capabilities
of making a software correlator using grids technologies.

\scarie\ is a typical example of a recent trend of the e-Science 
community in which the worldwide computation centers and the 
scientific instruments are connected through high-speed networks. We think that to 
generalize the utilization of such world-size inter-connected facility, 
the grids and their middlewares have to offer services that matches the three
following important aspects:
\begin{itemize}
\item \emph{performances}: solving bigger problem imply to use larger and/or
  more efficient hardware. 
\item \emph{isolated environment}: some users requires guarantee of isolated 
  execution environment in which the users'application cannot interfere. This is 
the case for real-time application or benchmarks.
\item \emph{scheduling}: may be needed when application requires to be synchronized
  with "external events": like a radio-telescope observing at specific date. 
\end{itemize}
In \scarie\ we are facing these three challenges, as we want to build a 
high-performance software correlator compatible with a real-time VLBI 
observation. These three aspects are hot-topics in the grid community,
especially the networking side; probably because worldwide networking (Internet) 
has a long history in being a \emph{best-effort} shared resourced. As best-effort 
is not compatible with isolated-environment we are conducting our 
experiments using the experimental \das3\ grids and its an
user-controllable dynamic photonic network called StarPlane that could permit 
us to build, on demand and application specific, isolated virtual-network 
on top of the complete grid. \\


The rest of the paper is organized as follows. Section~\ref{sec:vlbi}
contains a general introduction to VLBI and its recent development 
called \evlbi. Section~\ref{sec:softwarecorrelation} describes the 
architecture of the software correlator. Section~\ref{sec:network} 
describes how the software correlator make use of the StarPlane services 
and the \das3 resources. We present benchmarks in Section~\ref{sec:benchmarks} 
and conclude the paper with future work in Section~\ref{sec:conclusion}.




%%% Local Variables:
%%% mode: latex
%%% TeX-master: "Ingrid"
%%% End:
