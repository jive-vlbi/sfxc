\section{Execution and deployment.}
In section \ref{sec:introduction} are presented the two operational 
mode we target for the software correlator: normal and real-time. 
The first one correspond to a now rather standard: high performance computing 
application and thus can be executed relatively easily 
on top of existing grids middleware equipped with an 
MPI implementation. The real-time operational mode 
is much more challenging as we will discuss now. 

\subsection{Real-time and quality of service}
The term \emph{real-time} has a lot of meaning in the computer science community, 
in this text we will consider that a real-time computation is a computation in which 
\emph{the amount of buffering for an infinitely long experiment will only require 
a finite amount of buffers}. This imply that to be able to process the incoming 
data at the time they arrives and to have guarantee that once the application
is started the requested quality of service will be maintained by all
the service providers. 

Networking issues are one of the most interresting challenges 
of the SCARIe project. The first challenge is to deliver the 
data from the radio-telescope to the computation centrer. SCARIe
is running on a cluster-grid architecture in which computational 
resource are connected into a large networks. 

In its best scenario 
the amount of data transmitted to correlate the signal is of the same 
order than the one of the incoming signal. The consequence of that is that 
the grid have to be based on a relatively fast network.  

The regular Internet best-effort Layer3 IP routing has great
flexibility but is slow and unpredictable; on the other hand,
dedicated \textit{lightpaths} as available in \textit{lambda
  Grids}~\cite{eslea-2007}, with their predictable delays and
throughputs offer good performances and guarantee on the Quality of
Service (QoS). 


 (cpu, memory, networking, disk-space), to be 
able to \emph{acquire} a fixed amount of them and to have guarantee 
that the amount on which agreed SCARIe and the service provider 
will be respected. 
 
It is also noticeable that this complete isolation is made more and 
more available on usual desktop technologies...CFScheduler with groups 
in the Linux Kernel as well as Bandwidth-io control and Virtualization.
Different research grid are also investigating the field as in 
Grid'5000 that offer a complete node-isolation as well as extending the 
control of the user up to the ability to install their own task-dedicated 
operating system. 

\subsection{Running SCARIe on DAS-3 and Starplane}
The \das3\ supercomputer~\cite{das3} was deployed in the summer of
2007. It is distributed over 5 locations in the Netherlands and
connected by an photonic network called StarPlane. StarPlane is also a
research project funded by the Netherlands Organization for Scientific
Research (NWO) and carried out at two Dutch Universities: the
University of Amsterdam (UvA) and the Vrije Universiteit Amsterdam
(VU). The goals of the project is to build an
\textit{application-controlled photonic network}. 

StarPlane tries to fill the gap between the two
approaches by experiencing application-controlled dynamic photonic
network. The project's vision is that applications, having direct
control of the network paths, could make better use of the network
than if they were managed by a third party. Giving end user an access
to dedicated connection has been implemented in many of the current
research and education networks. The Dutch National Resarch and
Education network SURFnet is one of them. SURFnet6 deploys multiple
fiber optic rings that connect the academic and research locations
around The Netherlands. One of these rings connects the universities
in Leiden, Delft and Amsterdam, the locations of the \das3\ clusters.
Into this ring eight wavelengths constitute the StarPlane lightpaths.
The distinctive features of StarPlane in comparison with other similar
initiatives are the emphasis on fast reconfiguration times, in the
orders of seconds or sub-seconds, the use of photonic equipment in the
core, and the tigh coupling between the user application and the
offered networking service.

The Dutch grid, DAS-3 in which the StarPlane project tend
to deliver a complete nodes-to-nodes virtual network grid service over the 
complete grid domain. 

The ends nodes can thus be considered as a suitable environment to offer
guaranteed services that can be exposed to the end-user by the grid 
middlewares. The other important service from which we would like to have 
guaranteed service is the network. We can find in the literrature numerous 
solution to build network with guaranteed QoS: like 
dedicated network \cite{blahblah}, ipv4 layer-3 extensions \cite{}, ipv6 flow
control and routing \cite{}, token networks \cite{}. In SCARIe we are working 
closely with the StarPlane project that aim to deliver such on-demand 
virtual network as a grid service. The originality of the StarPlane project 
is in its attempt to build the virtual network at the lowest possible networking 
layer. This have the advantage to improve performances and to dramatically 
reduce the cost (the price or the energy) per byte transfered \cite{}. 
A complete virtual network for an application can then be build using only
vlan-on-mac layer 2 switches and layer-1 photonic lighpath. The other originality 
is that the photonic lightpath can dynamically reorganized to adapt to the
user-application requests. 


